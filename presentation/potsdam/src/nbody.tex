\begin{frame}
    \frametitle{The {\nbody} problem}
    \framesubtitle{Definition}

    Purely dynamic problem, in which the bodies orbital evolution
    is determined exclusive by the \blue{gravitational interaction},

    \begin{align}
        \bs{\ddot{r}}_{i} &= -G \sum\limits^{N}_{\substack{j=1\\j\neq i}}
                              m_{j} {(\bs{r}_i - \bs{r}_j)\over
                              | \bs{r}_i - \bs{r}_j|^{3}}\label{eq:nbody},
    \end{align}

    \noindent
    where $G$ is the gravitational constant
    ($6.67384\times 10^{-11} m^{3} kg^{-1} s^{-2}$),
    $m_j$ is the mass of the $j$th particle
    and $\bs{r}_j$ the position in \emph{Cartesian} coordinates.
    \begin{block}{\red{Note}}
        We denote vectors by bold fonts.
    \end{block}

\end{frame}


\begin{frame}
    \frametitle{The {\nbody} problem}
    \framesubtitle{Checking the system evolution}

    \begin{itemize}
        \item The \blue{initial condition} are usually the masses,
            position and velocity.

        \item \blue{Chaotic nature}, the evolution of this systems
         will depend of the initial parameters.

        \item The often invariant to check the integration of the system,
            is the system's \blue{energy},

            \begin{align}
                E &= {1 \over 2} \sum\limits^{N}_{i=1} m_{i} \bs{v}_{i}^{2} -
                     \sum\limits_{i=1}^{N} \sum\limits_{j > i}^{N}
                     {G m_{i} m_{j} \over |\bs{r}_{i} - \bs{r}_{j}|},
            \end{align}

            \noindent
            where $\bs{v}_i$ is the velocity of the particle $i$.
    \end{itemize}

\end{frame}


\begin{frame}
    \frametitle{The {\nbody} problem}
    \framesubtitle{Particle's time steps}

    \begin{itemize}
        \item The real scenario, \blue{individual} time steps.
        \begin{itemize}
            \item \red{Hard} scenario for parallel computing.
        \end{itemize}
        \item Forming groups of particles, \blue{block} time steps scheme~\cite{Press86}.
        \begin{itemize}
            \item  This time step scheme is popular among  {\nbody} code,
                like Starlab~\cite{portegies2001, hut2003}, Aarseth {\nbody}
                codes~\cite{Aarseth99, Aarseth03,NitadoriAarseth2012},
                $\phi$GRAPE~\cite{harfst2008},
                which gives us the possibility to check our algorithm behavior.
        \end{itemize}
    \end{itemize}
\end{frame}
